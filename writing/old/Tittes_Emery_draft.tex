\documentclass[11pt]{article}
\usepackage[sc]{mathpazo} %Like Palatino with extensive math support
\usepackage{amsmath}
\usepackage{fullpage}
\usepackage[authoryear,sectionbib,sort]{natbib}
%\usepackage{cite}
\linespread{1.7}
\usepackage[utf8]{inputenc}
\usepackage{lineno}

%%%%%%%%%%%%%%%%%%%%%
% LaTeX packages
%%%%%%%%%%%%%%%%%%%%%
% Please be sparing in your use of additional LaTeX packages, and
% upload any required style files to Editorial Manager with the file
% type "LaTeX ancillary files (.sty, .bst)."

%%%%%%%%%%%%%%%%%%%%%
% Line numbering
%%%%%%%%%%%%%%%%%%%%%
\usepackage{lineno}
% Please use line numbering with your initial submission and
% subsequent revisions. After acceptance, please comment out 
% the commands \usepackage{lineno}, \linenumbers{} 
% and \modulolinenumbers[3] below.

\title{Tolerance Curves \\
\textit{Lasthenia} Habitat \\
Bayesian Modeling and Comparitive Method
}

%%%%%%%%%%%%%%%%%%%%%
% Authorship
%%%%%%%%%%%%%%%%%%%%%
% Please remove authorship information while your paper is under review,
% unless you wish to waive your anonymity under double-blind review. 
% Remember to uncomment the information after acceptance.

\author{Silas B. Tittes$^{1,\ast}$ \\ 
Generic H. Collaborator$^{2}$ \\ 
Nancy C. Emery$^{1}$}

\date{}

\begin{document}

\maketitle

\noindent{}1. University of Colorado, Boulder, Colorado 80309;

\noindent{}2. University of BLANK;

\noindent{}$\ast$ Corresponding author; e-mail: silas.tittes@colorado.edu.

%\noindent{}$\ddag$ ORCIDs: Cook, 0000-0002-1234-5678; Collaborator, 0000-0001-5678-9123; Expert, 0000-0003-3456-789X.

\bigskip

\textit{Manuscript elements}: Figure~1, figure~2, table~1, online
appendices~A and B (including figure~A1 and figure~A2). Figure~2 is to
print in color.

\bigskip

\textit{Keywords}: Tolerance curves, Phenotypic Plasticity, Niche, Bayesian.

\bigskip

\textit{Manuscript type}: Article. 
% Or e-article, note, e-note, natural history miscellany,
% e-natural history miscellany, comment, reply, symposium, or
% countdown to 150.

\bigskip

\noindent{\footnotesize Prepared using the suggested \LaTeX{} 
template for \textit{Am.\ Nat.}}

\linenumbers{}
\modulolinenumbers[3]

\newpage{}

\section*{Abstract}


\newpage{}

\section*{Introduction}

% The journal does not have numbered sections in the main portion of
% articles. Please refrain from using section references such as
% section~\ref{section:CountingOwlEggs}, and refer to sections by name
% (e.g. section ``Counting Owl Eggs'').


% Please note that we prefer (\citealt{Xiao2015}) to \citep{Xiao2015},
% since \citep{} inserts a comma after "et al."

(\citealt{morrissey_variation_2016})

\section*{Methods}

\subsection*{study system}

\subsection*{experimental design}

\subsection*{bayesian modeling of tolerance curves}

\subsection*{phylogeny}

\subsection*{comparitive analyses}

(fig~\ref{Fig:test}). 

\section*{Results}

\section*{Discussion}

\section*{Conclusion}

%%%%%%%%%%%%%%%%%%%%%
% Acknowledgments
%%%%%%%%%%%%%%%%%%%%%
% You are encouraged to remove the Acknowledgments section while
% your paper is under review (unless you wish to waive your anonymity
% under double-blind review) if the Acknowledgments reveal your
% identity. If you remove this section, you will need to add it back
% in to your final files after acceptance.

%\section*{Acknowledgments}

%OEC would like to thank the world. GHC is much indebted to 
%the solar system. AQE was supported by a generous grant from 
%the Milky Way (MW/01010/987654).

\newpage{}

\renewcommand{\thesection}{\Alph{section}}

\section*{Online Appendix A: Supplementary Figures}

% Subsection numbering is permitted (but by no means necessary) in 
% online appendices. Please note that if you have sections (thus Online
% Appendix A, B, and C), these will become three separate online PDFs.
% You may wish to consolidate these into one PDF (hence one section,
% divided into subsections as necessary). Please reset counters for
% each such section. 

\renewcommand{\theequation}{A\arabic{equation}}
% redefine the command that creates the equation number.
\renewcommand{\thetable}{A\arabic{table}}
\setcounter{equation}{0}  % reset counter 
\setcounter{figure}{0}
\setcounter{table}{0}

\subsection*{Fox--dog encounters through the ages}

The quick red fox jumps over the lazy brown dog. The quick red fox has 
always jumped over the lazy brown dog. The quick red fox began jumping 
over the lazy brown dog in the 19th century and has never ceased from so 
jumping, as we shall see in 
%figure~\ref{Fig:Jumps}.

[Figure A1 goes here.]

[Figure A2 goes here.]

\newpage{}

\section*{Online Appendix B: Additional Methods}

\renewcommand{\theequation}{B\arabic{equation}}
% redefine the command that creates the equation number.
\setcounter{equation}{0}  % reset counter 
\renewcommand{\thetable}{B\arabic{table}}
\setcounter{figure}{0}
\setcounter{table}{0}

\subsection*{Measuring the height of fox jumps without a meterstick}

Pellentesque ac nibh placerat, luctus lectus non, elementum mauris. 
Morbi odio velit, eleifend ut hendrerit vitae, consequat sit amet 
nulla. Pellentesque porttitor vitae nisl quis tempus. Pellentesque 
habitant morbi tristique senectus et netus et malesuada fames ac 
turpis egestas. Praesent ut nisi odio. Vivamus vel lorem gravida 
odio molestie volutpat condimentum et arcu. 

\begin{equation}
{ \frac{1}{N_k-1} \sum \limits_{t=1}^{N_k} (M_{tjk} - \bar{M}_{jk})^2}
\end{equation}

\subsection*{Quantifying the brownness of the dog}

aliquam porta metus, quis malesuada orci faucibus quis. Suspendisse nunc 
magna, tristique sit amet sollicitudin nec, elementum et lacus. Sed 
vitae elementum mi. In hac habitasse platea dictumst. Etiam eu tortor 
elit. Sed ac tortor purus. Aliquam volutpat, odio sit amet posuere 
pretium, dolor ex interdum ante, sed luctus quam eros ac nulla. 

\begin{equation}
{ (\sum \limits_{p=1}^P {n_{sp}})^{-1}\sum \limits_{p=1}^P {n_{sp}Q_{p}}}
\end{equation}

\newpage{}

%%%%%%%%%%%%%%%%%%%%%
% Bibliography
%%%%%%%%%%%%%%%%%%%%%
% You can either type your references following the examples below, or
% compile your BiBTeX database and paste the contents of your .bbl file
% here. The amnatnat.bst style file should work for this---but please
% email the journal office at amnat at uchicago dot edu if you run into
% any hitches with it!
% The list below includes sample journal articles, book chapters, and
% Dryad references.

\bibliography{Tittes_tolerancecurves}

\bibliographystyle{amnatnat}

\newpage{}

\section*{Tables}
\renewcommand{\thetable}{\arabic{table}}
\setcounter{table}{0}

\begin{table}[h]
\caption{Animals in various cities with equations}
\label{Table:Okapi}
\centering
\begin{tabular}{llc}\hline
Animal    & City         & Equation \\ \hline
Dog       & Springfield  & $x+y=z$ \\
Fox       & Indianapolis & $2x+2y=2z$ \\
Okapi$^a$ & Chicago      & $x-y<z$ \\
Badger    & Madison      & $x+2y>z$ \\ \hline
\end{tabular}
\bigskip{}
\\
{\footnotesize Note: Table titles should be short. Further details 
should go in a `notes' area after the tabular environment, as shown 
here. $^a$ Okapis are not native to Chicago, but they are to be met with 
in both of the major Chicagoland zoos.}
\end{table}

\newpage{}

%%%%%%%%%%%%%%%%%%%%%
% Figure legends
%%%%%%%%%%%%%%%%%%%%%
% Please include all figure legends in a separate section at the end of 
% the document. If you use \label{} and \ref{} to refer to your figures,
% these can still work even if you comment out the %includegraphics{}
% line. If you refer to figures as "fig. 1" (etc.) manually, the
% legends can also appear simply as paragraphs.
% For submission, please upload the relevant figure files separately to
% Editorial Manager; Editorial Manager should insert them at the end of
% the PDF automatically.
% Figure legends should be concise, though they can be longer than the
% titles of tables.

\section*{Figure legends}

\begin{figure}[h!]
%\includegraphics{filename}
\caption{Figure 
legends should be concise, though they can be longer than the titles of 
tables.}
\label{Fig:test}
\end{figure}

%\subsection*{Online figure legends}

%\renewcommand{\thefigure}{A\arabic{figure}}
%\setcounter{figure}{0}

%\begin{figure}[h!]
%\includegraphics{jumps20m}
%\caption{\textit{A}, the quick red fox proceeding to jump 20~m straight 
%into the air over not one, but several lazy dogs. \textit{B}, the quick 
%red fox landing gracefully despite the skepticism of naysayers.}
%\label{Fig:Jumps}
%\end{figure}

%\begin{figure}[h!]
%\includegraphics{jumps-nr-okapi}
%\caption{The quicker the red fox jumps, the likelier it is to land near 
%\label{Fig:JumpsOk}
%\end{figure}

%\renewcommand{\thefigure}{B\arabic{figure}}
%\setcounter{figure}{0}

\end{document}
